%%%%%%%%%%%%%%%%%%%%%%%%%%%%%%%%%%%%%%%%%%%%%%%%%%%%%%%%%%%
% --------------------------------------------------------
% Rho
% LaTeX Template
% Version 2.1.1 (01/09/2024)
%
% Authors: 
% Guillermo Jimenez (memo.notess1@gmail.com)
% Eduardo Gracidas (eduardo.gracidas29@gmail.com)
% 
% License:
% Creative Commons CC BY 4.0
% --------------------------------------------------------
%%%%%%%%%%%%%%%%%%%%%%%%%%%%%%%%%%%%%%%%%%%%%%%%%%%%%%%%%%%

\documentclass[9pt,letterpaper,onecolumn]{rho-class/rho}
%\usepackage[english]{babel}

%% Spanish babel recomendation
\usepackage[spanish,es-nodecimaldot,es-noindentfirst]{babel}

\setbool{rho-abstract}{true} % Set false to hide the abstract
\setbool{corres-info}{true} % Set false to hide the corresponding author section

%----------------------------------------------------------
% TITLE
%----------------------------------------------------------

%\journalname{Example Template}
\title{DevGraph: Simulación de red social}

%----------------------------------------------------------
% AUTHORS AND AFFILIATIONS
%----------------------------------------------------------

\author[$\dagger$]{Franco Aguilar}
\author[$\dagger$]{Iván Mansilla}
\author[$\dagger$]{Diego Sanhueza}
\author[$\dagger$]{Duvan Figueroa}
\author[$\dagger$]{Miguel Maripillan}

%----------------------------------------------------------

%\affil[1]{Affiliation of author one}
%\affil[2]{Affiliation of author two}
%\affil[3]{Affiliation of author three}
\affil[$\dagger$]{Universidad de Magallanes}

%----------------------------------------------------------
% DATES
%----------------------------------------------------------

\dates{Informe fue compilado el día \today}

%----------------------------------------------------------
% FOOTER INFORMATION
%----------------------------------------------------------

%\leadauthor{Author last name et al.}
%\footinfo{Creative Commons CC BY 4.0}
\smalltitle{Estructuras de datos}
%\institution{Universidad de Magallanes}
\theday{\today} %\today

%----------------------------------------------------------
% ARTICLE INFORMATION
%----------------------------------------------------------

%\corres{Provide the corresponding author information and publisher here.}
%\email{example@organization.com.}
%\doi{\url{https://www.doi.org/exampledoi/XXXXXXXXXX}}

%\received{March 20, 2024}
%\revised{April 16, 2024}
%\accepted{April 20, 2024}
%\published{May 21, 2024}

%\license{Rho LaTeX Class \ccLogo\ This document is licensed under Creative Commons CC BY 4.0.}

%----------------------------------------------------------
% ABSTRACT
%----------------------------------------------------------

\begin{abstract}
    POR HACER
\end{abstract}

%----------------------------------------------------------

\keywords{C, Grafos, Red Social, Simulación, Estructuras de Datos}

%----------------------------------------------------------

\begin{document}
	
    \maketitle
    \thispagestyle{firststyle}
    \tableofcontents
    %\linenumbers

%----------------------------------------------------------

\section{Introducción}

    \rhostart{P}OR HACER

\section{Objetivo Principal}

    \rhostart{P}OR HACER

\section{Plantemiento del desarrollo del proyecto}

    \rhostart{P}OR HACER

\section{Implementación}

    \rhostart{P}OR HACER

\section{Gestión del equipo de trabajo}

    \rhostart{P}OR HACER

\section{Posibles mejoras}

    \rhostart{P}OR HACER

\section{Ejemplo de uso}

    \rhostart{P}OR HACER

\section{Conclusiones}

    \rhostart{P}OR HACER

    

\section{Rho packages}

        \begin{rhoenv}[frametitle=Environment with custom title]
            Hello! I am an example of the \textit{rhoenv} included in rhoenvs \LaTeX\ package. Here you can include relevant information or notes about your work. You can modify my title directly in the code.
        \end{rhoenv}

\section{Codes}

    This class\footnote{Hello there! I am a footnote :)} includes the \textit{listings} package, which offers customized features for adding codes specially for C, C++, \LaTeX\ and Matlab. You can customize the format in \textit{rho class} file.

    \lstinputlisting[caption=Example of matlab code., label={lst:listing-Mat}, language=Matlab]{example.m}

    
        
%----------------------------------------------------------

\section{Referencias}


%----------------------------------------------------------

\end{document}