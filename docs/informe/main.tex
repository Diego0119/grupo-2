%%%%%%%%%%%%%%%%%%%%%%%%%%%%%%%%%%%%%%%%%%%%%%%%%%%%%%%%%%%
% --------------------------------------------------------
% Rho
% LaTeX Template
% Version 2.1.1 (01/09/2024)
%
% Authors: 
% Guillermo Jimenez (memo.notess1@gmail.com)
% Eduardo Gracidas (eduardo.gracidas29@gmail.com)
% 
% License:
% Creative Commons CC BY 4.0
% --------------------------------------------------------
%%%%%%%%%%%%%%%%%%%%%%%%%%%%%%%%%%%%%%%%%%%%%%%%%%%%%%%%%%%

\documentclass[9pt,letterpaper,onecolumn]{rho-class/rho}
%\usepackage[english]{babel}

%% Spanish babel recomendation
\usepackage[spanish,es-nodecimaldot,es-noindentfirst]{babel}

\setbool{rho-abstract}{true} % Set false to hide the abstract
\setbool{corres-info}{true} % Set false to hide the corresponding author section

%----------------------------------------------------------
% TITLE
%----------------------------------------------------------

%\journalname{Example Template}
\title{DevGraph: Simulación de red social}

%----------------------------------------------------------
% AUTHORS AND AFFILIATIONS
%----------------------------------------------------------

\author[$\dagger$]{Franco Aguilar}
\author[$\dagger$]{Iván Mansilla}
\author[$\dagger$]{Diego Sanhueza}
\author[$\dagger$]{Duvan Figueroa}
\author[$\dagger$]{Miguel Maripillan}

%----------------------------------------------------------

%\affil[1]{Affiliation of author one}
%\affil[2]{Affiliation of author two}
%\affil[3]{Affiliation of author three}
\affil[$\dagger$]{Universidad de Magallanes}

%----------------------------------------------------------
% DATES
%----------------------------------------------------------

\dates{Informe fue compilado el día \today}

%----------------------------------------------------------
% FOOTER INFORMATION
%----------------------------------------------------------

%\leadauthor{Author last name et al.}
%\footinfo{Creative Commons CC BY 4.0}
\smalltitle{Estructuras de datos}
%\institution{Universidad de Magallanes}
\theday{\today} %\today

%----------------------------------------------------------
% ARTICLE INFORMATION
%----------------------------------------------------------

%\corres{Provide the corresponding author information and publisher here.}
%\email{example@organization.com.}
%\doi{\url{https://www.doi.org/exampledoi/XXXXXXXXXX}}

%\received{March 20, 2024}
%\revised{April 16, 2024}
%\accepted{April 20, 2024}
%\published{May 21, 2024}

%\license{Rho LaTeX Class \ccLogo\ This document is licensed under Creative Commons CC BY 4.0.}

%----------------------------------------------------------
% ABSTRACT
%----------------------------------------------------------

\begin{abstract}
    POR HACER
\end{abstract}

%----------------------------------------------------------

\keywords{C, Grafos, Red Social, Simulación, Estructuras de Datos}

%----------------------------------------------------------

\begin{document}
	
    \maketitle
    \thispagestyle{firststyle}
    \tableofcontents
    %\linenumbers

%----------------------------------------------------------

\section{Introducción}

    \rhostart{P}OR HACER

\section{Objetivo Principal}

    \rhostart{E}l objetivo principal del proyecto es crear una simulación de una red social eficiente y rápida, combinando armónicamente los distintos tipos de estructuras de datos vistos durante el curso de Estructuras de Datos (grafos, listas, hash tables, colas de prioridad, etc.).

    \subsection{Objetivos secundarios}
    \begin{itemize}
        \item Implementar algoritmos de búsqueda eficientes, priorizando el tiempo de respuesta
        \item Implementar buenas prácticas de codificación
        \item Reforzar habilidades de trabajo en equipo y coordinación de tareas entre pares
    \end{itemize}


\section{Planteamiento del desarrollo del proyecto}

    \rhostart{P}OR HACER

\section{Implementación}

    La implementación del proyecto fue realizada en lenguaje C, utilizando librerías estándar. El código fue escrito de manera modular, o sea, las funciones fueron separadas en diferentes archivos según su funcionalidad. Esto permitió una mayor manera de organización y mantenimiento del código, así como una mejor comprensión del funcionamiento del programa.

    \subsection{Estructura de directorios}
    En el directorio principal se encuentran los archivos \texttt{Makefile}, con el cual es posible compilar el programa de manera automática, y el archivo \texttt{README.MD}, que contiene información sobre el proyecto y su funcionamiento. Además, se encuentra el archivo \texttt{subtopics}, que contiene una lista de subtópicos modificable, que se utilizará para el funcionamiento de la aplicación.

    \vspace{0.5cm}

    Dentro del directorio \texttt{src} se encuentran los archivos fuente del programa, que son los siguientes:
    \begin{itemize}
        \item \texttt{main.c}: Contiene el flujo principal del programa.
        \item \texttt{database.c}: Contiene las funciones para la gestión y guardado de la base de datos.
        \item \texttt{graph.c}: Contiene las funciones para gestionar los grafos y relaciones entre usuarios.
        \item \texttt{hash\_table.c}: Contiene las funciones para gestionar tablas hash.
        \item \texttt{users.c}: Contiene las funciones para gestionar usuarios y publicaciones.
        \item \texttt{utilities.c}: Contiene las funciones de utilidad y testing para el programa.
        \item \texttt{heaps.c}: Contiene las funciones para gestionar las colas de prioridad.
    \end{itemize}

    Por otro lado, en el directorio \texttt{incs} se encuentran los archivos de cabeceras correspondiente a cada uno de los archivos fuente.

\section{Gestión del equipo de trabajo}

    \rhostart{E}l equipo de trabajo constó de 5 personas, uno de ellos designado por el profesor como líder del grupo, con la responsabilidad de coordinar el trabajo y la gestión del proyecto. Debido a la cantidad de personas involucradas, fue fundamental establecer mecanismos de comunicación, coordinación y orden para asegurar la colaboración y la coordinación de los diferentes miembros del equipo. Algunos de estos mecanismos fueron:

    \subsection{Normas de codificación}

        Normas de codificación fueron establecidas para el proyecto, con el objetivo de garantizar la coherencia y la consistencia en la codificación. Estas normas incluían:

        \begin{itemize}
            \item \textbf{Constantes}: Para las constantes se utilizó la convención de \texttt{SCREAMING\_SNAKE\_CASE}, con el objetivo de diferenciarlas de las variables y funciones.
            \item \textbf{Variables:} Para las variables comunes se utilizó la convención \texttt{camelCase}
            \item \textbf{Funciones:} Para las funciones se utilizó la convención \texttt{snake\_case}, así diferenciándose de las variables y constantes.
            \item \textbf{Tipos de datos:} Para los tipos de datos creados con \texttt{typedef} se utilizó la convención \texttt{PascalCase}, con el objetivo de diferenciarlos de los tipos de datos nativos de C.
            \item \textbf{Llaves de apertura:} Se estableció la convención de que las llaves de apertura se escribieran en la misma línea que el código que las contiene
        \end{itemize}

        \begin{lstlisting}[caption={Ejemplo de código, utilizando las normas de codificación}, label={lst:normas_codificacion}, language=C]
            #define MAX_CHAR 256 // Constante
            typedef _User *User; // Tipo de dato
            Graph newGraph; // Variable
            void print_all_users(Graph graph); // Funcion
        \end{lstlisting}

    \subsection{Comunicación}

    Un aspecto de suma importancia para el proyecto es la comunicación entre los miembros del equipo, ya que sin ella, no sería posible garantizar la colaboración y la coordinación en el desarrollo del proyecto. Debido al tamaño del equipo y la complejidad del proyecto esto fue un desafío, debido a las diferencias en horarios y responsabilidades académicas de cada miembro. Para resolver este problema fue escencial establecer un medio de comunicación remoto, que permitiera a cada uno comunicarse de manera rápida y en el lugar en que estuviese.

    Se creó un chat en línea con cada integrante del equipo, este chat utilizándose para discutir sobre el desarrollo del proyecto y su organización, así como para compartir información y recursos. 

    \vspace{0.5cm}

    Esto resultó exitoso, ya que las reuniones presenciales resultaron ser extremadamente dificiles de coordinar, la comunicación remota permitiendo a los miembros ver y discutir los detalles del proyecto en el momento del día en que estuviesen disponibles. Sin embargo, esto también implicó consecuencias, ya que dio paso a que fuera fácil que algún miembro del equipo "desapareciese"; sin embargo, estas situaciones fueron manejadas a través de la comunicación por privado con este para consultar su situación.

    \subsection{División del trabajo}

    Al inicio del proyecto se decidió establecer \textbf{objetivos a corto plazo} para cada miembro del equipo, según las estructuras de datos que se deseaban implementar en el proyecto. A cada uno se le asignaron según afinidad, capacidad y gusto propio.

    \vspace{0.5cm}

    Una vez las estructuras de datos hechas, se decidió crear una \textbf{lista de tareas} a realizar, cada miembro con al menos una tarea a realizar. Esta lista fue creada en base a las ideas discutidas por los integrantes del equipo, y se fue actualizando constantemente en función de las necesidades del proyecto.



\section{Posibles mejoras}

    \rhostart{P}OR HACER

\section{Ejemplo de uso}

    \rhostart{P}OR HACER

\section{Conclusiones}

    \rhostart{P}OR HACER


    
        
%----------------------------------------------------------

\section{Referencias}


%----------------------------------------------------------

\end{document}